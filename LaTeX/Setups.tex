\documentclass{article}
\usepackage{amsmath}
\usepackage[spanish]{babel}

\title{Setups de simulaciones}
\author{Pedro J. Romero Gombau}
\date{17/05/24}

\begin{document}

\maketitle

\section{Introducción}
En este documento se encuentran los diferentes setups que he ido utilizando para afinar las simulaciones de ANSYS que me han llevado a la solución final.
\section{Setup 1}
El primer setup utilizado es:
\begin{align*}
    T_{sim}=150ms~~~~~~T_{step}=5ms\to30~steps
    \\~\\
    I(t=0)=3.5~A\to I(t=50~ms)=I(t=T_{sim})=0~A
    \\~\\
    V_{coil}=13.1~V~~~~~~R_{coil}=3.65~\Omega
    \\~\\
    m_{bar}=0.019~kg~~~~~~v_{bar~ini}=0~ms^{-1}
\end{align*}

\section{Quadratic Equations}
The general form of a quadratic equation is $ax^2 + bx + c = 0$. The solution can be found using:
\[ x = \frac{-b \pm \sqrt{b^2 - 4ac}}{2a} \]

\section{Calculus}
Some basic calculus formulas include derivatives and integrals.
\subsection{Derivatives}
\begin{itemize}
\item $\frac{d}{dx}x^n = nx^{n-1}$
\item $\frac{d}{dx}\sin x = \cos x$
\item $\frac{d}{dx}\cos x = -\sin x$
\end{itemize}

\subsection{Integrals}
\begin{itemize}
\item $\int x^n \, dx = \frac{x^{n+1}}{n+1} + C$, for $n \neq -1$
\item $\int e^x \, dx = e^x + C$
\item $\int \cos x \, dx = \sin x + C$
\end{itemize}

\end{document}
