\section{Introducción}


En el ámbito de la ingeniería y la física aplicada, las \textit{coilguns}, también conocidas como \textit{lanzaderas electromagnéticas}, representan una tecnología de creciente interés debido a su potencial en aplicaciones tanto industriales como militares. El concepto de las lanzaderas electromagnéticas se originó en el siglo XIX, cuando se empezaron a explorar las propiedades del electromagnetismo y sus aplicaciones potenciales. De hecho, de esta época vviene uno de los nombres de. Sin embargo, fue en el siglo XX cuando estas ideas comenzaron a materializarse de manera más concreta, gracias a los avances en la tecnología de materiales y la electrónica. La necesidad de métodos de lanzamiento no explosivos en aplicaciones militares y aeroespaciales impulsó la investigación y el desarrollo de las lanzaderas electromagnéticas.

Las principales aplicaciones de las lanzaderas electromagnéticas se encuentran en el ámbito militar, donde se utilizan para el lanzamiento de proyectiles a alta velocidad sin la necesidad de explosivos químicos. Esta tecnología ofrece ventajas significativas, como la reducción del desgaste mecánico y la capacidad de ajustar la fuerza de lanzamiento con precisión. Además, en el sector aeroespacial, las lanzaderas electromagnéticas se consideran una alternativa prometedora para el lanzamiento de satélites y otros objetos al espacio, debido a su eficiencia energética y menor impacto ambiental en comparación con los cohetes tradicionales.

En la industria, las lanzaderas electromagnéticas se utilizan en procesos de manufactura que requieren la propulsión de materiales a altas velocidades. También se están explorando aplicaciones en el campo de la medicina, como en dispositivos de resonancia magnética y aceleradores de partículas para tratamientos médicos avanzados.

La investigación en lanzaderas electromagnéticas continúa evolucionando, con esfuerzos centrados en mejorar la eficiencia, la precisión y la viabilidad económica de estos dispositivos. A medida que la tecnología avanza, se espera que las lanzaderas electromagnéticas jueguen un papel cada vez más importante en diversas industrias, ofreciendo soluciones innovadoras y sostenibles para una amplia gama de aplicaciones.

El funcionamiento básico de una coilgun se basa en la creación de un campo magnético mediante el paso de una corriente eléctrica a través de una bobina de cobre. Cuando se aplica corriente a la bobina, se genera un campo magnético que ejerce una fuerza sobre el proyectil, generalmente una barra de material ferromagnético, a la que me referiré durante este proyecto como \textbf{vástago}. El proceso de aceleración comienza cuando la corriente eléctrica, controlada por un circuito electrónico, fluye a través de la bobina, creando un campo magnético que atrae el proyectil hacia el centro de la bobina. Antes de que los centros de la bobina y el vástago estén alineados, la corriente se corta, provocando que este último continúe su movimiento hacia adelante debido a su inercia.



%El objetivo principal de este trabajo de fin de grado es diseñar una práctica universitaria centrada en la optimización de los parámetros eléctricos y geométricos de una bobina para maximizar la fuerza y la velocidad de salida del proyectil. Para alcanzar este objetivo, se ha realizado un análisis electromagnético detallado de las ecuaciones que describen los fenómenos eléctricos en la bobina y su relación con los parámetros dinámicos del sistema. Este análisis no solo facilita la comprensión de los fundamentos teóricos del dispositivo, sino que también establece una base sólida para su optimización. Además, se ha desarrollado un prototipo funcional capaz de lanzar el proyectil y medir su velocidad, permitiendo así validar los resultados teóricos y prácticos.

%El desarrollo del proyecto se ha estructurado en varias fases. Inicialmente, se realizó un desarrollo analítico de las fórmulas necesarias para describir las interacciones entre los campos electromagnéticos de la bobina y sus efectos en el vástago. Posteriormente, se diseñó una geometría parametrizada utilizando el software ANSYS Maxwell, con el objetivo de llevar a cabo simulaciones transitorias. Estas simulaciones permiten estimar los valores dinámicos experimentados por el proyectil, proporcionando la flexibilidad de ajustar los parámetros geométricos y eléctricos para realizar experimentos virtuales. Finalmente, se construyó un prototipo funcional para validar los resultados teóricos y simulados.

%El presente documento se estructura de la siguiente manera: en primer lugar, se presenta una revisión de la literatura existente sobre lanzaderas y sus aplicaciones. A continuación, se describen los métodos y materiales utilizados en el desarrollo del proyecto, incluyendo el análisis teórico, las simulaciones en ANSYS Maxwell y la construcción del prototipo. Posteriormente, se analizan y discuten los resultados obtenidos y se compara la teoría con la práctica. Finalmente, se ofrecen conclusiones y recomendaciones para futuros trabajos en este campo.

%Este trabajo pretende servir como una herramienta educativa. Se espera que tanto el desarrollo teórico como las simulaciones proporcionen al departamento de Ingeniería Eléctrica de Tecnun la capacidad para realizar una práctica en la que los estudiantes puedan utilizar los modelos desarrollados en este trabajo para optimizar los parámetros del dispositivo. Esta práctica permitirá a los estudiantes aplicar conceptos teóricos en un entorno práctico, desarrollando habilidades analíticas y experimentales esenciales. Además, se pretende que los alumnos experimenten con diferentes configuraciones y parámetros, observando directamente cómo estos afectan al rendimiento de la lanzadera. Con este enfoque, se busca fomentar una comprensión más profunda de los principios electromagnéticos, al mismo tiempo que se dota a los estudiantes de las competencias necesarias para abordar problemas complejos en sus futuras carreras profesionales.