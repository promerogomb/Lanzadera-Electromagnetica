\addtocontents{toc}{\protect\setcounter{tocdepth}{-1}}
\section*{Resumen}

En las carreras de ingeniería, un aspecto muy enriquecedor son las prácticas, en las que los alumnos aplican los conocimientos aprendidos. Con este fin, en este PFG se ha propuesto el diseño de una lanzadera electromagnética, en la que los alumnos tendrán que poner en práctica sus habilidades. Para su diseño, se han realizado una serie de desarrollos teórico-prácticos.

Dichos desarrollos constan de uno analítico que ha proporcionado las fórmulas necesarias para generar una aplicación en MATLAB, un desarrollo mediante ANSYS Maxwell para realizar simulaciones con el sistema y un diseño de un prototipo funcional. Todo esto se ha materializado en una práctica para que los alumnos cursando Sistemas Eléctricos II puedan interactuar con el sistema y mejorar sus competencias en ingeniería eléctrica.

\vspace{0.5cm}
\textbf{Palabras clave:} Prácticas, Lanzadera electromagnética, MATLAB\textsuperscript{\textregistered}, ANSYS Maxwell, Prototipo.

\newpage
\thispagestyle{plain}
\addtocontents{toc}{\protect\setcounter{tocdepth}{-1}}
\section*{Abstract}
In engineering degrees, a highly enriching aspect is the practical sessions where students apply the knowledge they have learned. With this aim, this final degree project proposes the design of an electromagnetic launcher, in which students will need to put their skills into practice. For its design, a series of theoretical-practica l developments have been carried out.

These developments include an analytical one that has provided the necessary formulas to generate an application in MATLAB\textsuperscript{\textregistered}, a development using ANSYS Maxwell to perform simulations with the system, and the design of a functional prototype. All of this has been materialized into a practical session so that students taking Electrical Systems II can interact with the system and improve their competencies in electrical engineering.

\vspace{0.5cm}
\textbf{Key words:} Practicals, Coilgun,  MATLAB\textsuperscript{\textregistered}, ANSYS Maxwell, Prototipo.