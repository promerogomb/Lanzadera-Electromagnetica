% Paquetes
\documentclass[a4paper,12pt]{article}
\usepackage[utf8]{inputenc}
\usepackage{graphicx}
\usepackage{hyperref}
\usepackage{geometry}
\usepackage{ragged2e}
\usepackage{setspace}
\usepackage{anyfontsize}
\usepackage{tocloft}
\usepackage{titlesec}

% Preámbulo
\geometry{
  a4paper,         % Paper size
  left = 3cm,        % Left margin
  right = 3cm,       % Right margin
  top = 2.5cm,         % Top margin
  bottom = 2.5cm       % Bottom margin
}

%% Estilos de letras
\newcommand{\boldcenteredtext}[1]{
  \begin{center}
    \textbf{\fontsize{12pt}{14pt}\selectfont #1}
  \end{center}
}
\newcommand{\largeboldcenteredtext}[1]{
  \begin{center}
    \textbf{\fontsize{16pt}{18pt}\selectfont #1}
  \end{center}
}
\newcommand{\boldrightalignedtext}[1]{
  \begin{flushright}
    \textbf{\fontsize{12pt}{14pt}\selectfont #1}
  \end{flushright}
}
\newcommand{\centeredtext}[1]{
  \begin{center}
    \fontsize{12pt}{14pt}\selectfont #1
  \end{center}
}

%% Apariencia referencias
\hypersetup{
  colorlinks=true,
  linkcolor=black,
  urlcolor=blue,
  pdfborder={0 0 0}
}

%% Nombre de índices
\renewcommand{\contentsname}{Índice de contenidos
}
\renewcommand{\listfigurename}{Índice de figuras}
\renewcommand{\listtablename}{Índice de tablas}

% Documento
\begin{document}

\pagestyle{empty}

\begin{center}
  \includegraphics[width=0.4\linewidth, height=0.1\textheight]{FigurasMemoria/logoTecnun.png}
\end{center}

\vspace{1cm}

\boldcenteredtext{Proyecto Fin de Grado}

\largeboldcenteredtext{INGENIERÍA ELÉCTRICA}

\vspace{6cm}

\largeboldcenteredtext{Diseño y desarrollo de una lanzadera electromagnética}

\vspace{8cm}

\boldrightalignedtext{Pedro José Romero Gombau}
\boldrightalignedtext{Donostia-San Sebastián, mayo 2024}

\vspace{0.6cm}

\centeredtext{Po Manuel Lardizabal, 13. 20018 Donostia-San Sebastián, Gipuzkoa Tel. 943 219 877 · Fax 943 311 442 · www.tecnun.es}

\newpage
\thispagestyle{empty}
\mbox{}

% Indice títulos
\newpage
\thispagestyle{empty}
\tableofcontents

% Indice figuras
\newpage
\thispagestyle{empty}
\listoffigures

% Indice tablas
\newpage
\thispagestyle{empty}
\listoftables

\newpage
\thispagestyle{plain}
\section{Resumen}

Este trabajo de fin de grado trata acerca del diseño y la implementación de una lanzadora electromagnética, centrándose en el uso de ANSYS Maxwell para la simulación y el desarrollo de un prototipo funcional. Si bien el campo de la tecnología de las lanzadoras electromagnéticas está bien establecido, el objetivo principal de este proyecto es el diseño de una práctica universitaria en la que los alumnos dispongan de las fórmulas necesarias para optimizar la geometría y alimentación de la bobina y logren una mayor velocidad y fuerza de lanzamiento del proyectil. Los métodos empleados incluyen la creación de geometría en ANSYS Maxwell y simulaciones transitorias para analizar el comportamiento de la bobina, con énfasis en los parámetros dinámicos del proyectil. Además, se realizarán cálculos analíticos manuales para derivar relaciones electromagnéticas que rigen la interacción entre la bobina y el proyectil. En resumen, esta tesis presenta una exploración exhaustiva de las técnicas de diseño y simulación de una lanzadera electromagnética, con un enfoque en el aprendizaje de ANSYS Maxwell y la optimización de la geometría de la bobina para mejorar el rendimiento del proyectil. A pesar de la ausencia de contribuciones y resultados en esta etapa, el proyecto sirve como un valioso esfuerzo educativo en simulación electromagnética y desarrollo de prototipos.
\\~\\
\textbf{Palabras clave:}Lanzadera electromagnética, ANSYS Maxwell, Simulación, Prototipo, Optimización.

\newpage
\thispagestyle{plain}
\section{Abstract}
This undergraduate thesis focuses on the design and implementation of an electromagnetic launcher, emphasizing the use of ANSYS Maxwell for simulation and the development of a functional prototype. Although the field of electromagnetic launcher technology is well-established, the primary objective of this project is to design a university practical exercise in which students have the necessary formulas to optimize the geometry and power supply of the coil, achieving higher speed and force in projectile launch. The methods employed include creating geometry in ANSYS Maxwell and transient simulations to analyze the coil's behavior, with an emphasis on the dynamic parameters of the projectile. Additionally, manual analytical calculations will be conducted to derive electromagnetic relationships governing the interaction between the coil and the projectile. In summary, this thesis presents a comprehensive exploration of the design and simulation techniques for an electromagnetic launcher, focusing on learning ANSYS Maxwell and optimizing coil geometry to improve projectile performance. Despite the absence of contributions and results at this stage, the project serves as a valuable educational effort in electromagnetic simulation and prototype development.
\\~\\
\textbf{Key words:}Coilgun, ANSYS Maxwell, Simulation, Prototype, Optimization.

\newpage
\thispagestyle{plain}
\section{Introducción}

\newpage
\thispagestyle{plain}
\section{Motivación y objetivos}

\newpage
\thispagestyle{plain}
\section{Marco teórico}

\newpage
\thispagestyle{plain}
\section{Métodos}

\newpage
\thispagestyle{plain}
\section{Desarrollo}

\newpage
\thispagestyle{plain}
\subsection{Desarrollo teórico}

\newpage
\thispagestyle{plain}
\subsection{Simulaciones}

\newpage
\thispagestyle{plain}
\subsection{Prototipo}

\newpage
\thispagestyle{plain}
\section{Resultados}

\newpage
\thispagestyle{plain}
\section{Discusión y conclusiones}

\end{document}