% Paquetes
\documentclass[a4paper,11pt]{article}
\usepackage[utf8]{inputenc}
\usepackage{graphicx}
\usepackage{hyperref}
\usepackage{geometry}
\usepackage{ragged2e}
\usepackage{setspace}
\usepackage{anyfontsize}
\usepackage{tocloft}
\usepackage{titlesec}
\usepackage{parskip}
\usepackage{indentfirst}
\usepackage{textcomp}
\usepackage[spanish]{babel}
\usepackage{caption}
\usepackage[square,numbers]{natbib}
\usepackage{float}
\usepackage{enumitem}
\usepackage{amsmath}
\usepackage{amssymb}
\usepackage{matlab-prettifier}
\usepackage{xcolor}
\usepackage{array}


% Preámbulo
\geometry{
  a4paper,         % Paper size
  left = 3cm,        % Left margin
  right = 3cm,       % Right margin
  top = 2.5cm,         % Top margin
  bottom = 2.5cm       % Bottom margin
}

\setlength{\parindent}{2em}
\setlength{\parskip}{1em plus 0.5em minus 0.2em}
\captionsetup{font=small}
\addto\captionsspanish{
  \renewcommand{\tablename}{Tabla}
  \renewcommand{\listtablename}{Índice de tablas}
}

\spanishdecimal{.}

%% Definición código MATLAB:
\lstset{
  language=Matlab,
  basicstyle=\ttfamily\small,
  keywordstyle=\color{blue},
  commentstyle=\color{green},
  stringstyle=\color{red},
  numbers=left,
  numberstyle=\tiny\color{gray},
  stepnumber=1,
  numbersep=10pt,
  backgroundcolor=\color{white},
  showspaces=false,
  showstringspaces=false,
  showtabs=false,
  frame=single,
  tabsize=4,
  captionpos=b,
  breaklines=true,
  breakatwhitespace=false,
  title=\lstname,
  escapeinside={},
  morekeywords={},
}

%% Estilos de letras
\newcommand{\boldcenteredtext}[1]{
  \begin{center}
    \textbf{\fontsize{12pt}{14pt}\selectfont #1}
  \end{center}
}
\newcommand{\largeboldcenteredtext}[1]{
  \begin{center}
    \textbf{\fontsize{16pt}{18pt}\selectfont #1}
  \end{center}
}
\newcommand{\boldrightalignedtext}[1]{
  \begin{flushright}
    \textbf{\fontsize{12pt}{14pt}\selectfont #1}
  \end{flushright}
}
\newcommand{\centeredtext}[1]{
  \begin{center}
    \fontsize{10pt}{12pt}\selectfont #1
  \end{center}
}

%% Apariencia referencias
\hypersetup{
  colorlinks=true,
  linkcolor=black,
  urlcolor=blue,
  pdfborder={0 0 0},
  citecolor=black
}

%% Índices
\renewcommand{\contentsname}{Índice de contenidos}
\renewcommand{\listfigurename}{Índice de figuras}
\renewcommand{\cftfigpresnum}{\figurename\ }
\renewcommand{\cftfigaftersnum}{:}
\setlength{\cftfignumwidth}{3em}
\renewcommand{\listtablename}{Índice de tablas}

% Documento
\begin{document}

\pagestyle{plain}

\thispagestyle{empty}
\begin{center}
    \includegraphics[width=0.4\linewidth, height=0.1\textheight]{FigurasMemoria/logoTecnun.png}
  \end{center}
  
  \vspace{1cm}
  
  \boldcenteredtext{Proyecto Fin de Grado}
  
  \largeboldcenteredtext{INGENIERÍA ELÉCTRICA}
  
  \vspace{6cm}
  
  \largeboldcenteredtext{Diseño y desarrollo de una lanzadera electromagnética}
  
  \vspace{8cm}
  
  \boldrightalignedtext{Pedro José Romero Gombau}
  \boldrightalignedtext{Donostia-San Sebastián, mayo 2024}
  
  \vspace{0.6cm}
  
  \centeredtext{Po Manuel Lardizabal, 13. 20018 Donostia-San Sebastián, Gipuzkoa Tel. 943 219 877 · Fax 943 311 442 · www.tecnun.es}

\newpage
\thispagestyle{empty}
\mbox{}

\newpage
\addtocontents{toc}{\protect\setcounter{tocdepth}{-1}}
\section*{Resumen}

Este trabajo de fin de grado trata acerca del diseño y la implementación de una lanzadora electromagnética, centrándose en el uso de ANSYS Maxwell para la simulación y el desarrollo de un prototipo funcional. Si bien el campo de la tecnología de las lanzadoras electromagnéticas está bien establecido, el objetivo principal de este proyecto es el diseño de una práctica universitaria en la que los alumnos dispongan de las fórmulas necesarias para optimizar la geometría y alimentación de la bobina y logren una mayor velocidad y fuerza de lanzamiento del proyectil. Los métodos empleados incluyen la creación de geometría en ANSYS Maxwell y simulaciones transitorias para analizar el comportamiento de la bobina, con énfasis en los parámetros dinámicos del proyectil. Además, se realizarán cálculos analíticos manuales para derivar relaciones electromagnéticas que rigen la interacción entre la bobina y el proyectil. En resumen, esta tesis presenta una exploración exhaustiva de las técnicas de diseño y simulación de una lanzadera electromagnética, con un enfoque en el aprendizaje de ANSYS Maxwell y la optimización de la geometría de la bobina para mejorar el rendimiento del proyectil.
\\~\\
\textbf{Palabras clave:}Lanzadera electromagnética, ANSYS Maxwell, Simulación, Prototipo, Optimización.

\newpage
\thispagestyle{plain}
\addtocontents{toc}{\protect\setcounter{tocdepth}{-1}}
\section*{Abstract}
This undergraduate thesis focuses on the design and implementation of an electromagnetic launcher, emphasizing the use of ANSYS Maxwell for simulation and the development of a functional prototype. Although the field of electromagnetic launcher technology is well-established, the primary objective of this project is to design a university practical exercise in which students have the necessary formulas to optimize the geometry and power supply of the coil, achieving higher speed and force in projectile launch. The methods employed include creating geometry in ANSYS Maxwell and transient simulations to analyze the coil's behavior, with an emphasis on the dynamic parameters of the projectile. Additionally, manual analytical calculations will be conducted to derive electromagnetic relationships governing the interaction between the coil and the projectile. In summary, this thesis presents a comprehensive exploration of the design and simulation techniques for an electromagnetic launcher, focusing on learning ANSYS Maxwell and optimizing coil geometry to improve projectile performance.
\\~\\
\textbf{Key words:}Coilgun, ANSYS Maxwell, Simulation, Prototype, Optimization.

% Indice títulos
\newpage
\thispagestyle{empty}
\tableofcontents

% Indice figuras
\newpage
\thispagestyle{empty}
\listoffigures

% Indice tablas
\newpage
\thispagestyle{empty}
\listoftables

\addtocontents{toc}{\protect\setcounter{tocdepth}{3}}

\newpage
\section{Introducción}


En el ámbito de la ingeniería y la física aplicada, las \textit{coilguns}, también conocidas como \textit{lanzaderas electromagnéticas}, representan una tecnología de creciente interés debido a su potencial en aplicaciones tanto industriales como militares. El concepto de las lanzaderas electromagnéticas se originó en el siglo XIX, cuando se empezaron a explorar las propiedades del electromagnetismo y sus aplicaciones potenciales. De hecho, de esta época vviene uno de los nombres de. Sin embargo, fue en el siglo XX cuando estas ideas comenzaron a materializarse de manera más concreta, gracias a los avances en la tecnología de materiales y la electrónica. La necesidad de métodos de lanzamiento no explosivos en aplicaciones militares y aeroespaciales impulsó la investigación y el desarrollo de las lanzaderas electromagnéticas.

Las principales aplicaciones de las lanzaderas electromagnéticas se encuentran en el ámbito militar, donde se utilizan para el lanzamiento de proyectiles a alta velocidad sin la necesidad de explosivos químicos. Esta tecnología ofrece ventajas significativas, como la reducción del desgaste mecánico y la capacidad de ajustar la fuerza de lanzamiento con precisión. Además, en el sector aeroespacial, las lanzaderas electromagnéticas se consideran una alternativa prometedora para el lanzamiento de satélites y otros objetos al espacio, debido a su eficiencia energética y menor impacto ambiental en comparación con los cohetes tradicionales.

En la industria, las lanzaderas electromagnéticas se utilizan en procesos de manufactura que requieren la propulsión de materiales a altas velocidades. También se están explorando aplicaciones en el campo de la medicina, como en dispositivos de resonancia magnética y aceleradores de partículas para tratamientos médicos avanzados.

La investigación en lanzaderas electromagnéticas continúa evolucionando, con esfuerzos centrados en mejorar la eficiencia, la precisión y la viabilidad económica de estos dispositivos. A medida que la tecnología avanza, se espera que las lanzaderas electromagnéticas jueguen un papel cada vez más importante en diversas industrias, ofreciendo soluciones innovadoras y sostenibles para una amplia gama de aplicaciones.

El funcionamiento básico de una coilgun se basa en la creación de un campo magnético mediante el paso de una corriente eléctrica a través de una bobina de cobre. Cuando se aplica corriente a la bobina, se genera un campo magnético que ejerce una fuerza sobre el proyectil, generalmente una barra de material ferromagnético, a la que me referiré durante este proyecto como \textbf{vástago}. El proceso de aceleración comienza cuando la corriente eléctrica, controlada por un circuito electrónico, fluye a través de la bobina, creando un campo magnético que atrae el proyectil hacia el centro de la bobina. Antes de que los centros de la bobina y el vástago estén alineados, la corriente se corta, provocando que este último continúe su movimiento hacia adelante debido a su inercia.



%El objetivo principal de este trabajo de fin de grado es diseñar una práctica universitaria centrada en la optimización de los parámetros eléctricos y geométricos de una bobina para maximizar la fuerza y la velocidad de salida del proyectil. Para alcanzar este objetivo, se ha realizado un análisis electromagnético detallado de las ecuaciones que describen los fenómenos eléctricos en la bobina y su relación con los parámetros dinámicos del sistema. Este análisis no solo facilita la comprensión de los fundamentos teóricos del dispositivo, sino que también establece una base sólida para su optimización. Además, se ha desarrollado un prototipo funcional capaz de lanzar el proyectil y medir su velocidad, permitiendo así validar los resultados teóricos y prácticos.

%El desarrollo del proyecto se ha estructurado en varias fases. Inicialmente, se realizó un desarrollo analítico de las fórmulas necesarias para describir las interacciones entre los campos electromagnéticos de la bobina y sus efectos en el vástago. Posteriormente, se diseñó una geometría parametrizada utilizando el software ANSYS Maxwell, con el objetivo de llevar a cabo simulaciones transitorias. Estas simulaciones permiten estimar los valores dinámicos experimentados por el proyectil, proporcionando la flexibilidad de ajustar los parámetros geométricos y eléctricos para realizar experimentos virtuales. Finalmente, se construyó un prototipo funcional para validar los resultados teóricos y simulados.

%El presente documento se estructura de la siguiente manera: en primer lugar, se presenta una revisión de la literatura existente sobre lanzaderas y sus aplicaciones. A continuación, se describen los métodos y materiales utilizados en el desarrollo del proyecto, incluyendo el análisis teórico, las simulaciones en ANSYS Maxwell y la construcción del prototipo. Posteriormente, se analizan y discuten los resultados obtenidos y se compara la teoría con la práctica. Finalmente, se ofrecen conclusiones y recomendaciones para futuros trabajos en este campo.

%Este trabajo pretende servir como una herramienta educativa. Se espera que tanto el desarrollo teórico como las simulaciones proporcionen al departamento de Ingeniería Eléctrica de Tecnun la capacidad para realizar una práctica en la que los estudiantes puedan utilizar los modelos desarrollados en este trabajo para optimizar los parámetros del dispositivo. Esta práctica permitirá a los estudiantes aplicar conceptos teóricos en un entorno práctico, desarrollando habilidades analíticas y experimentales esenciales. Además, se pretende que los alumnos experimenten con diferentes configuraciones y parámetros, observando directamente cómo estos afectan al rendimiento de la lanzadera. Con este enfoque, se busca fomentar una comprensión más profunda de los principios electromagnéticos, al mismo tiempo que se dota a los estudiantes de las competencias necesarias para abordar problemas complejos en sus futuras carreras profesionales.

\newpage
\subsection{Motivación}

Trataré en este subapartado las motivaciones que han impulsado este proyecto y justifican el área de estudio del mismo. Tras haber llevado a cabo el desarrollo de la lanzadera electromagnética, he concluido que las motivaciones de este trabajo de final de grado son las siguientes: 

\begin{enumerate}
    \item \textbf{Innovación Tecnológica:} La investigación y desarrollo en tecnologías como la tratada en este trabajo representan una oportunidad para estar a la vanguardia en el campo de la ingeniería electromagnética. Este proyecto permite explorar y comprender los principios fundamentales del electromagnetismo aplicados a un sistema real y funcional.
    \item \textbf{Aplicación de Conocimientos Teóricos:} La creación de una \textit{lanzadera} requiere la aplicación de conocimientos avanzados en física, matemáticas e ingeniería eléctrica. Este proyecto proporciona un contexto práctico en el que tanto yo como alumno, como los futuros estudiantes que lo utilicen, emplearán teorías y conceptos aprendidos en el aula para fortalecer su entendimiento de los fenómenos electromagnéticos a un nivel visual y palpable.
    \item \textbf{Desarrollo de Competencias Técnicas:} La construcción de la \textit{lanzadera} involucra diversas habilidades técnicas, desde el diseño y simulación en software especializado hasta la fabricación y prueba de placas electrónicas y prototipos funcionales. Este proceso mejora significativamente las competencias prácticas en el laboratorio, una habilidad esencial para cualquier ingeniero eléctrico.
    \item \textbf{Fomento de la Innovación Educativa:} El desarrollo de este proyecto no solo busca aportar al conocimiento técnico, sino también servir como una herramienta educativa innovadora. La práctica universitaria diseñada a partir de este proyecto permitirá a los estudiantes experimentar directamente con la optimización de parámetros electromagnéticos, desarrollando habilidades críticas y fomentando una mentalidad innovadora.
\end{enumerate}

Con esto queda justificada la realización de este proyecto de fin de grado, y podemos empezar a desarrollar el proceso de creación de la \textbf{lanzadera electromagnética}.

\newpage

\subsection{Objetivos y métodos}

Exploraremos ahora los principales objetivos del proyecto, desglosando cada parte constituyente y su resultado. Como ya se ha dicho en la introducción, el principal propósito es la creación de una práctica universitaria que se pueda realizar durante el primer o segundo curso, con la idea de atraer a nuevos ingenierios hacia el campo de la electricidad. Para lograr este objetivo principal, el trabajo se dividirá en tres partes: desarrollo teórico, simulaciones y desarrollo de un prototipo. Los objetivos y resultados esperados de cada parte son:

\begin{enumerate}
    \item \textbf{Desarrollo teórico:} Este apartado tiene como objetivo explorar las fórmulas que describen el comportamiento del vástago dentro de la bobina cuando es alimentada con corriente continua. El desarrollo resultará en una serie de fórmulas que constituirán un modelo del sistema, así como un programa que las implemente en una aplicación de \textit{MatLAB\textregistered}.
    \item \textbf{Simulaciones:} Las simulaciones tienen como objetivo obtener otro modelo físico del sistema, utilizando el método de los elementos finitos a través del software \textit{ANSYS Maxwell\textregistered}. El resultado esperado es un modelo paramétrico que permita introducir los valores de la geometría de la bobina y su alimentación, y devuelva los valores dinámicos del vástago. Se espera que estos resultados sean más precisos que los obtenidos mediante el desarrollo teórico y se pretende probar diferentes configuraciones hasta llegar a la más óptima.
    \item \textbf{Prototipo:} Esta parte tiene como objetivo el diseño y desarrollo de un prototipo funcional de lanzadera que permita comparar los resultados teóricos con los físicos. Será necesario diseñar un circuito electrónico de control con \textit{Arduino\textregistered} y un medio físico para sujetar y alimentar la bobina. El resultado esperado es un prototipo manejable y modular, con el cual se puedan probar diferentes configuraciones.
\end{enumerate}

\newpage
\section{Marco teórico}

En esta sección se desarrollará el funcionamiento teórico de las lanzaderas electromagnéticas, con un enfoque especial en los principios físicos que permiten su operación y la alta eficiencia en el lanzamiento de proyectiles. Comenzaremos con el análisis de los circuitos eléctricos y magnéticos equivalentes del dispositivo, presentando las relaciones principales entre sus componentes. Además, se explicará el circuito de control típico utilizado en estas aplicaciones.

\begin{figure}[h]
    %\centering \raggedleft \raggedright
    \includegraphics[width=\linewidth]{FigurasMemoria/fig3electromagnet.jpg}
    \caption{Campo magnético en una bobina. www.lanl.gov}
    \label{fig:3} %Para referenciar -> \ref{fig:figNum}
\end{figure}

Como se ha mencionado en la introducción, el funcionamiento de una lanzadera electromagnética se basa en la capacidad de las bobinas de generar un flujo magnético cuando se les aplica una corriente, como se puede ver en la. Esto es debido a la ley de Àmpere, que enuncia lo siguiente:

\begin{quote}
    Según la ley de Ampère, la integral de línea del campo magnético \(\mathbf{B}\) alrededor de un lazo cerrado es igual a \(\mu_0\) multiplicado por la corriente total \(I_{enc}\) que pasa a través de cualquier superficie delimitada por el lazo. Matemáticamente, esto se expresa como:
    \[
    \oint_{\partial S} \mathbf{B} \cdot d\mathbf{l} = \mu_0 I_{enc}
    \]
    donde \(\mu_0\) es la permeabilidad del vacío.
\end{quote}

La manera más básica de representar una lanzadera es un simple circuito con una fuente de alimentación, un interruptor controlado por un circuito electrónico y una bobina, que es la encargada de generar el campo.

\newpage
\section{Desarrollo}
\label{sec:desarrollo}

En esta sección se detallarán los pasos llevados a cabo para la consecución del objetivo principal de este proyecto: el diseño y construcción de una \textit{lanzadera electromagnética}. En los siguientes apartados, se describirán concisamente los procedimientos, herramientas y resultados obtenidos en cada una de estas fases del desarrollo.
\subsection{Desarrollo teórico}
\label{subsec:desarrollo}
Para comenzar el desarrollo teórico, primero definiremos la geometría de la bobina de manera esquemática para entender bien el sistema con el que trabajamos. De manera descriptiva, lo que tenemos es un cilindro hueco de radio \( r_{cint} \) y altura \( h_c \) sobre el cual enrollaremos un hilo de cobre \textbf{\textit{N}} veces, resultando en un radio exterior \( r_{cext} \). Ligeramente introducido en el cilindro hueco, se encuentra el vástago, que es un cilindro de acero (REFERENCIAR EL ACERO BIEN) de radio \( r_b \) y longitud \( L_b \). La corriente de alimentación será \( i_{dc} \). En la siguiente figura se resume todo en un esquema:

\begin{figure}[H]
    \centering
    \includegraphics[width=6cm]{FigurasMemoria/fig2esquemaGeom.png}
    \caption{Esquema de la geometría de la bobina. Elaboración propia.}
    \label{fig:3} %Para referenciar -> \ref{fig:figNum}
\end{figure}

Antes de empezar con el desarrollo, hay que aclarar que cuando se empezó este proyecto, se partió de una bobina ya hecha, con la que se han realizado todas las pruebas que se pueden ver en los siguientes apartados. Con el propósito de obtener valores numéricos en esta sección para comprobar de manera cualitativa si el modelo tiene sentido, utilizaremos también los datos de la geometría de la bobina de pruebas para calcular el resultado de las ecuaciones obtenidas. Dichos datos son:

\[
L_b=0.096m~~~~r_b=0.003045m
\\~\\
h_c=0.05321m~~~~r_{cext}=0.01064m
\\~\\
i_{dc}=3.5A~~N=500
\]

El desarrollo teórico empezará por buscar el valor de la inducción electromagnética en la barra, ya que con su valor podremos calcular la fuerza utilizando Lorentz (\textbf{COMPROBAR ESTO IGUAL HAY QUE PONER OTRA COSA}). Para ello, aplicaremos la \textbf{ley integral de Àmpere}:

\[
Ni=\oint{\vec{H}\vec{dl}}
\]

Para simplificar los cálculos, vamos a asumir un flujo uniforme en la bobina ya que tan solo en los extremos (\( \delta \)) el flujo magnético se curvará. Con esto, podemos escribir:

\[
Ni=Hh_c\to H=\frac{Ni_{dc}}{h_c}
\]

Buscamos ahora una expresión para la densidad de flujo:

\[
B=\mu_0\mu_r H=\mu_0\mu_r\frac{Ni_{dc}}{h_c}
\]

\newpage
\section{Cálculo por Elementos Finitos de la fuerza de atracción}
\label{sec:simulaciones}

En esta sección, se describe el proceso de creación del modelo de la lanzadera electromagnética realizado utilizando ANSYS Maxwell\textsuperscript{\textregistered}. Este software de métodos finitos está especializado en el análisis de sistemas electromagnéticos. El objetivo de estas simulaciones es obtener una comprensión detallada del comportamiento del campo magnético generado por la bobina y su interacción con el proyectil ferromagnético en la lanzadera electromagnética. Se ha dividido el proceso en tres subapartados: creación de la geometría, simulaciones instantáneas y simulaciones transitorias. En la primera, se mostrará el proceso de creación de la geometría paramétrica en 2D y 3D, en la segunda y tercera se explicarán las condiciones de contorno y resultados de las simulaciones instantáneas y transitorias, respectivamente. En el Anexo III se detalla el procedimiento de creación de la geometría y realización de simulaciones.

\subsection{Geometría}
El primer paso para realizar las simulaciones fue la creación de la geometría del sistema en ANSYS Maxwell\textsuperscript{\textregistered}. Para ello, se definieron las dimensiones y características de la bobina y el proyectil de manera paramétrica. Esto quiere decir que las dimensiones de los polígonos que conforman el modelo no son fijas, si no que están asociadas a variables por lo que simular diferentes configuraciones se convierte en algo más sencillo. La geometría fue primeramente diseñada en 3D, pero también se obtuvo la sección central del archivo tridimensional para poder simular en 2D, lo que permite realizar las simulaciones con un menor tiempo de computación. Una vez creada la geometría, se definieron los diferentes materiales y sus propiedades electromagnéticas. El material utilizado para los conductores del solenoide es cobre recocido, PVC para el soporte de la bobina y acero F114 para el vástago. Con esto, se modela la geometría a continuación mostrada:

\begin{itemize}
    \item \textbf{Geometría en 3D}:
    \begin{figure}[H]
        \centering
        \includegraphics[width=7cm]{FigurasMemoria/geom3d.png}
        \caption{Geometría de la barra y bobina en 3D.}
        \label{fig:geom3d} %Para referenciar -> \ref{fig:figNum}
    \end{figure}
    \item \textbf{Mallado en 3D}:
    \begin{figure}[H]
        \centering
        \includegraphics[width=7cm]{FigurasMemoria/mesh3d.png}
        \caption{Mallado de la barra y bobina en 3D.}
        \label{fig:mesh3d} %Para referenciar -> \ref{fig:figNum}
    \end{figure}
    \begin{figure}[H]
        \centering
        \includegraphics[width=9cm]{FigurasMemoria/meshDetail3d.png}
        \caption{Detalle del mallado de la barra y bobina en 3D.}
        \label{fig:meshDetail3d} %Para referenciar -> \ref{fig:figNum}
    \end{figure}

    \vspace{5cm}
    
    \item \textbf{Geometría en 2D}:
    \begin{figure}[H]
        \centering
        \includegraphics[width=9cm]{FigurasMemoria/BarGeom.png}
        \caption{Geometría de la barra y bobina en 2D.}
        \label{fig:BarGeom} %Para referenciar -> \ref{fig:figNum}
    \end{figure}
    \item \textbf{Mallado en 2D}
    \begin{figure}[H]
        \centering
        \includegraphics[width=9cm]{FigurasMemoria/BarGeomMesh.png}
        \caption{Mallado de la barra y bobina en 2D.}
        \label{fig:BarGeomMesh} %Para referenciar -> \ref{fig:figNum}
    \end{figure}
    \begin{figure}[H]
        \centering
        \includegraphics[width=9cm]{FigurasMemoria/BarMeshDetail.png}
        \caption{Detalle del mallado de la barra y bobina en 2D.}
        \label{fig:GeomMeshDetail} %Para referenciar -> \ref{fig:figNum}
    \end{figure}
\end{itemize}

La bobina (cilindro ocre), se ha generado de la manera mostrada y no con un polígono helicoidal debido a que el orden de magnitud de espiras (\(10^2\)) y el orden de magnitud de las medidas (\(10^{-3}\)) convierte esto en una muy compleja tarea, ya que ANSYS Maxwell\textsuperscript{\textregistered} no es un entorno de diseño en 3D. Al crear un cilindro con dimensiones apropiadas, se puede indicar al software que lo trate como un solenoide, asignándole un número de espiras y una corriente de excitación. El programa distribuye uniformemente el campo generado por la corriente en todo el volumen, lo que resulta en una aproximación muy válida de un solenoide, sin necesidad de crear una geometría extremadamente compleja \citep[p. 13]{ansoft2012maxwell}.

\subsection{Simulaciones instantáneas}
ANSYS Maxwell\textsuperscript{\textregistered} permite realizar simulaciones de diferentes tipos, de las cuales tienen interés para este proyecto las magnetoestáticas y las transitorias. Como primera prueba para el modelo de la figura \ref{fig:geom3d}, se realizó una simulación electroestática con una \(I_{cc}=3.65~A\) para poder observar los campos magnéticos del sistema, resultando en la siguiente figura:

\begin{figure}[H]
    \centering
    \includegraphics[width=14cm]{FigurasMemoria/fields8.PNG}
    \caption{Visualización de los campos magnéticos con la bobina energizada con \(I_{cc}=3.5~\text{A}\) y \(x=8/10 * h_c\).}
    \label{fig:fields8} %Para referenciar -> \ref{fig:figNum}
\end{figure}

Vemos que se puede apreciar perfectamente la forma descrita en las figuras \ref{fig:integralampere} y \ref{fig:electromagnet}, curvándose en los extremos del sistema pero permaneciendo bastante uniforme a lo largo de \(h_c\). Se muestra a continuación un detalle de los campos únicamente dentro de la bobina con la barra en la posición de la figura \ref{fig:fields8}:

\begin{figure}[H]
    \centering
    \includegraphics[width=3cm]{FigurasMemoria/fieldsDetail.jpg}
    \caption{Visualización de los campos magnéticos dentro del solenoide.}
    \label{fig:fieldsDetail} %Para referenciar -> \ref{fig:figNum}
\end{figure}

Además de visualizar los campos, ANSYS Maxwell\textsuperscript{\textregistered} permite añadir distintas medidas a la simulación, entre las que se encuentra la fuerza de atracción magnética. Realizando la simulación de nuevo para la configuración en 3D, se obtiene que la fuerza es igual a:

\begin{table}[h]
    \centering
    \begin{tabular}{|c|c|c|c|c|}
        \hline
        & \(F_x\) & \(F_y\) & \(F_z\) & \(F_{tot}\) \\
        \hline
        \(F_{ANSYS}\) & Data 2 & Data 3 & Data 4 & Data 5 \\
        \hline
    \end{tabular}
    \caption{Fuerzas de atracción magnética en los diferentes ejes del espacio.}
    \label{tab:fuerzas}
\end{table}

\subsection{Simulaciones transitorias}
Una vez observados los campos de manera estacionaria, el siguiente paso es conseguir la variación de los valores de fuerza con la posición del vástago, además sin forzar este movimiento. Es decir, especificar la región de movimiento y los valores de alimentación de la bobina y dejar que ANSYS Maxwell\textsuperscript{\textregistered} genere el movimiento y nos de información acerca de la fuerza y velocidad experimentadas por el vástago. Al realizar esto con el modelo 3D, ANSYS Maxwell\textsuperscript{\textregistered} devolvía un error fatal probablemente debido a que los ordenadores en los que se estaba realiando la simulación no eran lo suficientemente potentes. Se recurrió entonces a la utilización exclusiva de la geometría en 2D, la que dio resultados tanto de movimiento como fuerza. Para llegar a los últimos resultados, se realizaron varias simulaciones con varias configuraciones temporales diferentes, las cuales se muestran a continuación:
\subsection*{Configuración 1}
La primera configuración transitoria utilizada es:
\[
T_{sim}=150~ms \quad T_{step}=5~ms \to 30~steps
\]
\[
I(t=0)=3.5~A \quad I(t\geq 50~ms)=0~A
\]
\[
V_{coil}=13.1~V \quad R_{coil}=3.65~\Omega
\]
\[
m_{bar}=0.019~kg \quad v_{bar~ini}=0~ms^{-1}
\]
Las gráficas resultantes de fuerza-corriente y velocidad-posición son:
\begin{figure}[H]
    \centering
    \includegraphics[width=13cm]{FigurasMemoria/S1CurrentForce.jpg}
    \caption{Fuerza (verde) y corriente (rojo) en función del tiempo de la configuración 1.}
    \label{fig:S1CurrentForce} %Para referenciar -> \ref{fig:figNum}
\end{figure}
\begin{figure}[H]
    \centering
    \includegraphics[width=10cm]{FigurasMemoria/S1SpeedPosition.jpg}
    \caption{Posición (verde) y velocidad (rojo) en función del tiempo de en la configuración 1.}
    \label{fig:S1SpeedPosition} %Para referenciar -> \ref{fig:figNum}
\end{figure}
Como se puede observar en las figuras \ref{fig:S1CurrentForce} y \ref{fig:S1SpeedPosition}, \(T_{sim}\) es muy largo ya que la mayor parte del tiempo no ocurre nada. También es destacable que existen oscilaciones, lo que indica que la bobina está siendo alimentada durante demasiado tiempo y está siendo retenida en el centro.


\subsection*{Configuración 2}
La segunda configuración transitoria utilizada es:
\[
T_{sim}=75~ms \quad T_{step}=0.6~ms \to 125~steps
\]
\[
I(t=0)=3.5~A \quad I(t\geq 30~ms)=0~A
\]
\[
V_{coil}=13.1~V \quad R_{coil}=3.65~\Omega
\]
\[
m_{bar}=0.019~kg \quad v_{bar~ini}=0~ms^{-1}
\]
Las gráficas resultantes de fuerza-corriente y velocidad-posición son:
\begin{figure}[H]
    \centering
    \includegraphics[width=13cm]{FigurasMemoria/S2ForceCurrent.jpg}
    \caption{Fuerza (rojo) y corriente (verde) en función del tiempo de la configuración 2.}
    \label{fig:S2ForceCurrent} %Para referenciar -> \ref{fig:figNum}
\end{figure}
\begin{figure}[H]
    \centering
    \includegraphics[width=13cm]{FigurasMemoria/S2SpeedPosition.jpg}
    \caption{Posición (verde) y velocidad (rojo) en función del tiempo de la configuración 2.}
    \label{fig:S2SpeedPosition} %Para referenciar -> \ref{fig:figNum}
\end{figure}
En la segunda configuración, se utilizó un timestep muy pequeño para poder observar con precisión el movimiento del vástago. Se observa más ajustado el tiempo de simulación, aunque el tiempo de alimentación de la bobina sigue siendo muy elevado ya que el vástago sigue oscilando, aunque ya solo lo hace una vez. 

\subsection*{Configuración 3}
La tercera configuración transitoria utilizada es:
\[
T_{sim}=25~ms \quad T_{step}=1~ms \to 25~steps
\]
\[
I(t=0)=3.5~A \quad I(t\geq 15~ms)=0~A
\]
\[
V_{coil}=13.1~V \quad R_{coil}=3.65~\Omega
\]
\[
m_{bar}=0.019~kg \quad v_{bar~ini}=0~ms^{-1}
\]
Las gráficas resultantes de fuerza-corriente y velocidad-posición son:
\begin{figure}[H]
    \centering
    \includegraphics[width=13cm]{FigurasMemoria/S3ForceCurrent.jpg}
    \caption{Fuerza (rojo) y corriente (verde) en función del tiempo de la configuración 3.}
    \label{fig:S3ForceCurrent} %Para referenciar -> \ref{fig:figNum}
\end{figure}
\begin{figure}[H]
    \centering
    \includegraphics[width=13cm]{FigurasMemoria/S3SpeedPos.jpg}
    \caption{Posición (verde) y velocidad (rojo) en función del tiempo de la configuración 3.}
    \label{fig:S3SpeedPos} %Para referenciar -> \ref{fig:figNum}
\end{figure}

Esta configuración proporciona la evolución de la fuerza según el sistema que queremos crear, dejando de alimentar la bobina antes de que los centros estén alineados y permitiendo que el vástago siga avanzando por inercia, lo cual podemos comprobar observando la gráfica en la figura \ref{fig:S3SpeedPos}, en la que la posición no decrece en ningún momento. Sin embargo, los resultados de la magnitud de la fuerza no son verosímiles, pues son 40 veces mayores (\(F_{max}\approx 80~N\)) a la fuerza obtenida en la prueba de referencia (figura \ref{fig:dinamometro}).

Con esto se dará por finalizado el apartado de simulaciones. Aunque se ha encontrado una configuración que imita correctamente el comportamiento del sistema, los resultados obtenidos no son precisos en términos de magnitud. Dado que no se ha logrado avanzar en la mejora de la precisión de los resultados, se ha decidido no continuar dedicando más tiempo a afinar esta configuración. Se analizará en detalle lo obtenido en el apartado de validación de modelos (\ref{sec:resultados}).

\newpage
\subsection{Prototipo}
\label{subsec:prototipo}

\newpage
\section{Resultados}

\newpage
\section{Discusión y conclusiones}

\newpage
\begin{flushleft}
  \bibliographystyle{abbrvnat}
  \bibliography{referencias}
\end{flushleft}

\newpage
\section*{Anexo I. Código de MATLAB para la calculadora de fuerza.}
\label{sec:anexo1}
En este anexo se presenta el código escrito para la calculadora expuesta en el apartado de desarrollo teórico \ref{subsec:desarrolloteorico}

\newpage
\section*{Anexo II. Configuraciones transitorias.}
En este anexo se muestran las diferentes condiciones de inicio de las simulaciones transitorias. 




\end{document}