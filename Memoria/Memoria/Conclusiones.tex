\section{Conclusiones}
\label{sec:conclusiones}

Este proyecto de fin de grado se ha centrado en el diseño, simulación y desarrollo de un prototipo de lanzadera electromagnética. A lo largo del documento, se han abordado diferentes aspectos teóricos y prácticos necesarios para comprender y optimizar el funcionamiento de una lanzadera electromagnética. A continuación, se presentan las conclusiones más relevantes de cada sección:

\begin{itemize}
    \item \textbf{Cálculo analítico de la fuerza de atracción}: Se han establecido las bases teóricas necesarias para el entendimiento del comportamiento de las lanzaderas electromagnéticas. Desde la ley integral de Ampère y el análisis de circuitos magnéticos, se ha desarrollado un modelo teórico que permite calcular la fuerza de atracción magnética en función de los parámetros geométricos y eléctricos de la bobina y el vástago. Este modelo teórico ha sido implementado en MATLAB\textsuperscript{\textregistered}, facilitando el cálculo y la visualización de los resultados. El resultado de esta implementación ha sido positivo, con un error de tan solo el 12\%. Se considera, por tanto, un modelo válido para realizar predicciones de la fuerza de atracción magnética a la que será sometido un proyectil que entre en un sistema análogo al desarrollado.

    \item \textbf{Cálculo por elementos finitos de la fuerza de atracción}: Utilizando el software ANSYS Maxwell\textsuperscript{\textregistered}, se han realizado simulaciones tanto instantáneas como transitorias para analizar el comportamiento del sistema en diferentes condiciones. Aunque las simulaciones han proporcionado una buena comprensión del comportamiento del campo magnético y la fuerza de atracción, se ha identificado una limitación en la precisión de los resultados en términos de magnitud. Sin embargo, las simulaciones han sido útiles para confirmar el modelo de MATLAB\textsuperscript{\textregistered} y para observar cómo se comportan los campos magnéticos del sistema.

    \item \textbf{Desarrollo de una lanzadera electromagnética}: Se ha diseñado y construido un prototipo funcional de lanzadera electromagnética que complementa a los modelos teóricos de los apartados anteriores. Los resultados experimentales obtenidos con el prototipo han corroborado en gran medida las predicciones teóricas.
    
    \item \textbf{Práctica propuesta}: Uno de los objetivos principales de este proyecto ha sido el desarrollo de una práctica universitaria para los estudiantes de ingeniería eléctrica que cursen la asignatura de Sistemas Eléctricos I. El proyecto ha demostrado ser una excelente plataforma para la aplicación práctica de conocimientos teóricos en un contexto real, fomentando el interés por la ingeniería eléctrica y permitiendo que los alumnos mejoren sus aptitudes prácticas en el laboratorio. La práctica propuesta también proporciona una oportunidad para que los estudiantes experimenten con la optimización de parámetros y vean directamente los efectos de sus ajustes en el rendimiento de un sistema.
\end{itemize}

En resumen, este trabajo ha logrado combinar con éxito teoría, simulación y práctica para el diseño y desarrollo de una lanzadera electromagnética. Los resultados obtenidos en los diferentes ámbitos del proyecto validan cada uno de ellos, proporcionando una solución teórico-práctica general del sistema, el cual ha sido abordado de manera efectiva y desde varias perspectivas. Lo que este trabajo deja pendiente de cara al futuro es la optimización del modelo de elementos finitos, que ha fallado a la hora de proporcionar resultados magnitudinalmente precisos. Se deja al lector acceso completo a todo el desarrollo interno de este documento, así como de cada uno de los apartados, código y planteamientos del proyecto en \href{URL}{https://github.com/promerogomb/Lanzadera-Electromagnetica/}