\section*{Anexo II. Código de control de disparo.}

En este anexo se presenta el código de control de disparo de la lanzadera, escrito en Arduino:

\begin{lstlisting}[language=C, label={lst:arduino}]
    // Asignacion de pines
    const int relayPin = 11;
    const int sensor1Pin = 5;
    const int sensor2Pin = 4;
    const int buttonPin = 6;
    
    // Variables de control de tiempo
    unsigned long timeSensor1 = 0;
    unsigned long timeSensor2 = 0;
    bool sensor1HighDetected = false;
    
    void setup() {
      Serial.begin(9600);
    
      // Modos de los pines
      pinMode(relayPin, OUTPUT);
      pinMode(sensor1Pin, INPUT);
      pinMode(sensor2Pin, INPUT);
      pinMode(buttonPin, INPUT_PULLUP);
    
      digitalWrite(relayPin, LOW);
    }
    
    void loop() {
      // Comprobacion de pulsacion del boton
      if (digitalRead(buttonPin) == LOW) { // Condicion de pulsacion
        digitalWrite(relayPin, HIGH); // Activacion del rele
        int timeOn = millis(); // Tiempo de inicio de alimentacion
        // Reinicio de variables
        sensor1HighDetected = false;  
        timeSensor1 = 0;
        timeSensor2 = 0;
    
        // Esperamos a que el sensor a la salida de la bobina detecte el vastago
        while (!sensor1HighDetected) {
          if (digitalRead(sensor1Pin) == HIGH) {
            timeSensor1 = millis();       // Valor temporal
            sensor1HighDetected = true;
          }
        }
    
        digitalWrite(relayPin, LOW); // Apagamos el rele
        int timeOff = millis(); // Tiempo de corte de alimentacion
    
        // Esperamos a que el segundo sensor detecte el vastago
        while (timeSensor2 == 0) {
          if (digitalRead(sensor2Pin) == HIGH) {
            timeSensor2 = millis();   // Valor temporal
          }
        }
    
        // Calculamos las diferencias temporales
        unsigned long timeDifference = timeSensor2 - timeSensor1; // Intervalo de tiempo para la velocidad
        unsigned long currentTime = timeOff - timeOn; // Intervalo de tiempo de alimentacion
    
        // Mostramos los valores en el terminal
        Serial.println("Diferencia de tiempo de medicion: ");
        Serial.println(timeDifference);
        Serial.println(" milisegundos.");
        Serial.println("La bobina ha estado alimentada por: ");
        Serial.println(currentTime);
        Serial.println(" milisegundos.");
    
        delay(500);
      }
    }
    
\end{lstlisting}
