\section{Objetivos y motivación del proyecto}
\label{sec:motivacionyobjetivos}

\subsection{Objetivos y métodos}
Exploraremos ahora los principales objetivos del proyecto, desglosando cada parte constituyente y su resultado esperado. El principal propósito de este trabajo es el diseño de una práctica universitaria que se pueda realizar durante la asignatura de sistemas eléctricos, con la idea de atraer a nuevos ingenieros hacia el campo de la electricidad.

Para lograr este objetivo principal, el trabajo se dividirá en cuatro partes: desarrollo teórico, simulaciones, desarrollo de un prototipo y desarrollo de la práctica. Los objetivos y resultados esperados de cada parte son:

\begin{enumerate}
    \item \textbf{Desarrollo teórico:} Este apartado tiene como objetivo explorar las fórmulas que describen el comportamiento del vástago dentro de la bobina cuando es alimentada con corriente continua. El desarrollo resultará en una serie de fórmulas que constituirán un modelo del sistema, así como un programa que las implemente en una aplicación de \textit{MatLAB\textregistered}.
    \item \textbf{Simulaciones:} Las simulaciones tienen como objetivo obtener otro modelo teórico del sistema, utilizando el método de los elementos finitos a través del software \textit{ANSYS Maxwell\textregistered}. El resultado esperado es un modelo paramétrico que permita introducir los valores de la geometría de la bobina y su alimentación, y devuelva los valores dinámicos del vástago. Se espera que estos resultados sean más precisos que los obtenidos mediante el desarrollo teórico.
    \item \textbf{Prototipo:} Esta parte tiene como objetivo el diseño y desarrollo de un prototipo funcional de lanzadera que permita comparar los resultados teóricos con los físicos. Será necesario diseñar un circuito electrónico de control con \textit{Arduino\textregistered} y un medio físico para sujetar y alimentar la bobina. El resultado esperado es un prototipo manejable y modular, con el cual se puedan probar diferentes configuraciones.
    \item \textbf{Desarrollo de la práctica:} Con los resultados obtenidos en los apartados anteriores, se desarrollará un documento que presente el diseño de una bobina para implementar en el prototipo, a resolver por los alumnos que realicen la práctica. Se especificarán las variables que pueden modificar, así como las relaciones entre ellas, definidas a partir de las ecuaciones de electromagnetismo que gobiernan los circuitos magnéticos.
\end{enumerate}

\newpage
\subsection{Motivación}

El desarrollo de este proyecto está justificado por los siguientes puntos, que van a ser las principales áreas de influencia de este trabajo de final de grado:

\begin{enumerate}
    \item \textbf{Vanguardia Tecnológica:} La investigación y desarrollo en tecnologías como la tratada en este trabajo representan una oportunidad para estar a la vanguardia en el campo de la ingeniería electromagnética. Este proyecto permite explorar y comprender los principios fundamentales del electromagnetismo aplicados a un sistema real y funcional.
    \item \textbf{Aplicación de Conocimientos Teóricos:} La creación de una \textit{lanzadera electromagnética} requiere la aplicación de conocimientos avanzados en física, matemáticas e ingeniería eléctrica. Este proyecto proporciona un contexto práctico en el que tanto el autor como alumno, como los futuros estudiantes que lo utilicen, emplearán teorías y conceptos aprendidos en las aulas para fortalecer su entendimiento de los fenómenos electromagnéticos a un nivel visual y palpable.
    \item \textbf{Desarrollo de Competencias Técnicas:} La construcción de la \textit{lanzadera} involucra diversas habilidades técnicas, desde el diseño y simulación en software especializado hasta la fabricación y prueba de placas electrónicas y prototipos funcionales. Este proceso mejora significativamente las competencias prácticas en el laboratorio, una habilidad esencial para cualquier ingeniero eléctrico.
    \item \textbf{Fomento de la Innovación Educativa:} El desarrollo de este proyecto no solo busca aportar al conocimiento técnico, sino también servir como una herramienta educativa innovadora. La práctica universitaria diseñada a partir de este proyecto permitirá a los estudiantes experimentar directamente con la optimización de parámetros electromagnéticos, desarrollando habilidades críticas y fomentando una mentalidad innovadora.
\end{enumerate}

Con esto queda justificada la realización de este proyecto de fin de grado, y podemos empezar a desarrollar el proceso de creación de la \textbf{lanzadera electromagnética}.