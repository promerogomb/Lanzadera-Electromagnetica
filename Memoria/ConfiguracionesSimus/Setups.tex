\documentclass{article}
\usepackage{amsmath}
\usepackage[spanish]{babel}
\usepackage{graphicx}

\title{Setups de simulaciones}
\author{Pedro J. Romero Gombau}
\date{17/05/24}

\begin{document}

\maketitle

\section{Introducción}
En este documento se encuentran los diferentes setups que he ido utilizando para afinar las simulaciones de ANSYS que me han llevado a la solución final.
\section{Setup 1}
El primer setup utilizado es:
\begin{align*}
    T_{sim}=150ms~~~~~~T_{step}=5ms\to30~steps
    \\~\\
    I(t=0)=3.5~A\to I(t=50~ms)=I(t=T_{sim})=0~A
    \\~\\
    V_{coil}=13.1~V~~~~~~R_{coil}=3.65~\Omega
    \\~\\
    m_{bar}=0.019~kg~~~~~~v_{bar~ini}=0~ms^{-1}
\end{align*}

\section{Setup 2}
El primer setup utilizado es:
\begin{align*}
    T_{sim}=150ms~~~~~~T_{step}=5ms\to30~steps
    \\~\\
    I(t=0)=3.5~A\to I(t=50~ms)=I(t=T_{sim})=0~A
    \\~\\
    V_{coil}=13.1~V~~~~~~R_{coil}=3.65~\Omega
    \\~\\
    m_{bar}=0.019~kg~~~~~~v_{bar~ini}=0~ms^{-1}
\end{align*}

\section{Setup 3}
El tercer setup utilizado es:
\begin{align*}
    T_{sim}=40ms~~~~~~T_{step}=2ms\to20~steps
    \\~\\
    I(t=0)=3.5~A\to I(t>15~ms)=0~A
    \\~\\
    V_{coil}=13.1~V~~~~~~R_{coil}=3.65~\Omega
    \\~\\
    m_{bar}=0.019~kg~~~~~~v_{bar~ini}=0~ms^{-1}
\end{align*}
Este setup da problemas con la magnitud de la fuerza. Es posible que no se esté analizando bien el comportamiento del vástago porque el timestep es muy grande.

\section{Setup 3}
El cuarto setup utilizado es:
\begin{align*}
    T_{sim}=25ms~~~~~~T_{step}=1ms\to25~steps
    \\~\\
    I(t=0)=3.5~A\to I(t>15~ms)=0~A
    \\~\\
    V_{coil}=13.1~V~~~~~~R_{coil}=3.65~\Omega
    \\~\\
    m_{bar}=0.019~kg~~~~~~v_{bar~ini}=0~ms^{-1}
\end{align*}
\end{document}
