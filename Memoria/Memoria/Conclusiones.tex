\section{Conclusiones}
\label{sec:conclusiones}

Este proyecto de fin de grado se ha centrado en el diseño, simulación y desarrollo de un prototipo de lanzadera electromagnética. A lo largo del documento, se han abordado diferentes aspectos técnicos y teóricos necesarios para comprender y optimizar el funcionamiento de este tipo de dispositivos. A continuación, se presentan las conclusiones más relevantes:

\begin{itemize}
    \item \textbf{Cálculo analítico de la fuerza de atracción}: Se han establecido las bases teóricas necesarias para el entendimiento del comportamiento de las lanzaderas electromagnéticas. A partir de la ley de Ampère y el análisis de circuitos magnéticos, se ha desarrollado un modelo teórico que permite calcular la fuerza de atracción magnética en función de los parámetros geométricos y eléctricos de la bobina y el vástago. Este modelo teórico ha sido implementado en un programa en Matlab que facilita el cálculo y la visualización de los resultados.
    \item \textbf{Cálculo por elementos finitos de la fuerza de atracción}: Utilizando el software ANSYS Maxwell, se han realizado simulaciones tanto instantáneas como transitorias para analizar el comportamiento del sistema en diferentes condiciones. Aunque las simulaciones han proporcionado una buena comprensión del comportamiento del campo magnético y la fuerza de atracción, se ha identificado una limitación en la precisión de los resultados en términos de magnitud. Sin embargo, las simulaciones han sido útiles para validar el modelo teórico y ajustar los parámetros del sistema.
    \item \textbf{Desarrollo de una lanzadera electromagnética}: Se ha diseñado y construido un prototipo funcional de lanzadera electromagnética que permite la validación experimental de los modelos teóricos y las simulaciones. El prototipo ha demostrado ser una herramienta eficaz para la enseñanza y la investigación en el campo de la ingeniería electromagnética. Los resultados experimentales obtenidos con el prototipo han corroborado en gran medida las predicciones teóricas y han proporcionado datos valiosos para futuras mejoras y optimizaciones del diseño.
    \item \textbf{Práctica propuesta}: Uno de los objetivos principales de este proyecto ha sido el desarrollo de una práctica universitaria para estudiantes de ingeniería. El proyecto ha demostrado ser una excelente plataforma para la aplicación práctica de conocimientos teóricos en un contexto real, fomentando el interés por la ingeniería electromagnética y mejorando las habilidades técnicas y analíticas de los estudiantes.
\end{itemize}

En resumen, este trabajo ha logrado combinar con éxito teoría, simulación y práctica para el diseño y desarrollo de una lanzadera electromagnética. Los resultados obtenidos son prometedores y abren la puerta a futuras investigaciones y desarrollos en este campo, tanto en el ámbito académico como en aplicaciones industriales y tecnológicas.