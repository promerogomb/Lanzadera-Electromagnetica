\addtocontents{toc}{\protect\setcounter{tocdepth}{-1}}
\section*{Resumen}

Este trabajo de fin de grado trata acerca del diseño y la implementación de una lanzadora electromagnética, centrándose en el uso de ANSYS Maxwell para la simulación y el desarrollo de un prototipo funcional. Si bien el campo de la tecnología de las lanzadoras electromagnéticas está bien establecido, el objetivo principal de este proyecto es el diseño de una práctica universitaria en la que los alumnos dispongan de las fórmulas necesarias para optimizar la geometría y alimentación de la bobina y logren una mayor velocidad y fuerza de lanzamiento del proyectil. Los métodos empleados incluyen la creación de geometría en ANSYS Maxwell y simulaciones transitorias para analizar el comportamiento de la bobina, con énfasis en los parámetros dinámicos del proyectil. Además, se realizarán cálculos analíticos manuales para derivar relaciones electromagnéticas que rigen la interacción entre la bobina y el proyectil. En resumen, esta tesis presenta una exploración exhaustiva de las técnicas de diseño y simulación de una lanzadera electromagnética, con un enfoque en el aprendizaje de ANSYS Maxwell y la optimización de la geometría de la bobina para mejorar el rendimiento del proyectil.
\\~\\
\textbf{Palabras clave:}Lanzadera electromagnética, ANSYS Maxwell, Simulación, Prototipo, Optimización.

\newpage
\thispagestyle{plain}
\addtocontents{toc}{\protect\setcounter{tocdepth}{-1}}
\section*{Abstract}
This undergraduate thesis focuses on the design and implementation of an electromagnetic launcher, emphasizing the use of ANSYS Maxwell for simulation and the development of a functional prototype. Although the field of electromagnetic launcher technology is well-established, the primary objective of this project is to design a university practical exercise in which students have the necessary formulas to optimize the geometry and power supply of the coil, achieving higher speed and force in projectile launch. The methods employed include creating geometry in ANSYS Maxwell and transient simulations to analyze the coil's behavior, with an emphasis on the dynamic parameters of the projectile. Additionally, manual analytical calculations will be conducted to derive electromagnetic relationships governing the interaction between the coil and the projectile. In summary, this thesis presents a comprehensive exploration of the design and simulation techniques for an electromagnetic launcher, focusing on learning ANSYS Maxwell and optimizing coil geometry to improve projectile performance.
\\~\\
\textbf{Key words:}Coilgun, ANSYS Maxwell, Simulation, Prototype, Optimization.